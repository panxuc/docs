\documentclass{article}
\usepackage{geometry}
\usepackage{ctex}
\usepackage{tikz}
\usetikzlibrary{arrows, calc, positioning}
\geometry{a4paper, landscape, scale = 0.9}
\pagestyle{empty}

\title{detailAdd}
\author{Xuc Pan}
\date{July 2023}

\tikzstyle{intt}=[draw,text centered,minimum size=6em,text width=5.25cm,text height=0.34cm]
\tikzstyle{intl}=[draw,text centered,minimum size=2em,text width=2.75cm,text height=0.34cm]
\tikzstyle{int}=[draw,minimum size=2.5em,text centered,text width=3.5cm]
\tikzstyle{intg}=[draw,minimum size=3em,text centered,text width=6.cm]
\tikzstyle{sum}=[draw,shape=circle,inner sep=2pt,text centered,node distance=3.5cm]
\tikzstyle{summ}=[drawshape=circle,inner sep=4pt,text centered,node distance=3.cm]

\begin{document}
 \begin{figure}[!htb]
  \centering
  \begin{tikzpicture}[>=latex', auto]
   \node [intl] (kp) {添加学生成绩信息};
   \node [intl] (ki1) [node distance = 4cm, right of = kp] {选择添加学生};
   \node [intl] (ki4) [node distance = 4cm, right of = ki1] {检查该课程是否存在于数据库中};
   \node [intl] (ki3) [node distance = 2cm, above of = ki4] {输入要添加的课程名};
   \node [intl] (ki2) [node distance = 2cm, above of = ki3] {输入要添加的学生信息};
   \node [intl] (ki5) [node distance = 4cm, right of = ki4] {输入课程信息};
   \node [intl] (ki6) [node distance = 2cm, below of = ki4] {输入课程成绩};
   \node [intl] (ki7) [node distance = 2cm, below of = ki6] {询问是否继续输入};

   \draw[->] (kp) -- (ki1);
   \draw[->] (ki1) |- (ki2);
   \draw[->] (ki2) -- (ki3);
   \draw[->] (ki3) -- (ki4);
   \draw[->] (ki4) -- (ki5);
   \draw[->] (ki4) -- (ki6);
   \draw[->] (ki6) -- (ki7);
   \draw[->] (ki5) |- (ki6);
  \end{tikzpicture}
 \end{figure}
\end{document}