\documentclass{article}
\usepackage{geometry}
\usepackage{ctex}
\usepackage{tikz}
\usetikzlibrary{arrows, calc, positioning}
\geometry{a4paper, landscape, scale = 0.9}
\pagestyle{empty}

\title{detail2}
\author{Xuc Pan}
\date{July 2023}

\tikzstyle{intt}=[draw,text centered,minimum size=6em,text width=5.25cm,text height=0.34cm]
\tikzstyle{intl}=[draw,text centered,minimum size=2em,text width=2.75cm,text height=0.34cm]
\tikzstyle{int}=[draw,minimum size=2.5em,text centered,text width=3.5cm]
\tikzstyle{intg}=[draw,minimum size=3em,text centered,text width=6.cm]
\tikzstyle{sum}=[draw,shape=circle,inner sep=2pt,text centered,node distance=3.5cm]
\tikzstyle{summ}=[drawshape=circle,inner sep=4pt,text centered,node distance=3.cm]

\begin{document}
 \begin{figure}[!htb]
  \centering
  \begin{tikzpicture}[>=latex', auto]
   \node [int] (kp) {学生成绩管理系统};
   \node [intl] (ki1) [node distance = 2cm, below of = kp] {登录界面};
   \node [intl] (ki5) [node distance = 2cm, below of = ki1] {查询学生成绩信息};
   \node [intl] (ki4) [node distance = 3.5cm, left of = ki5] {删除学生成绩信息};
   \node [intl] (ki3) [node distance = 3.5cm, left of = ki4] {修改学生成绩信息};
   \node [intl] (ki2) [node distance = 3.5cm, left of = ki3] {添加学生成绩信息};
   \node [intl] (ki6) [node distance = 3.5cm, right of = ki5] {排序学生成绩信息};
   \node [intl] (ki7) [node distance = 3.5cm, right of = ki6] {打印学生成绩信息};
   \node [intl] (ki8) [node distance = 3.5cm, right of = ki7] {保存与载入学生成绩信息};

   \draw[->] (kp) -- (ki1);
   \draw[->] (ki1) -| (ki2);
   \draw[->] (ki1) -| (ki3);
   \draw[->] (ki1) -| (ki4);
   \draw[->] (ki1) -- (ki5);
   \draw[->] (ki1) -| (ki6);
   \draw[->] (ki1) -| (ki7);
   \draw[->] (ki1) -| (ki8);
  \end{tikzpicture}
 \end{figure}
\end{document}